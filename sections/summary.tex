\section{Project Summary}
\subsection{Document Overview} 
% Each proposal must contain a summary of the proposed project not more than {\bf one page in length}. The Project
% Summary consists of an overview, a statement on the intellectual merit of the proposed activity, and a statement
% on the broader impacts of the proposed activity.
% The overview includes a description of the activity that would result if the proposal were funded and a statement
% of objectives and methods to be employed.
% 
% The Project Summary should be written in the third person, informative to other persons working in
% the same or related fields, and, insofar as possible, understandable to a scientifically or technically 
% literate lay reader. It should not be an abstract of the proposal.
% 
% If the Project Summary contains special characters it may be uploaded as a Supplementary Document.
% {\bf Project Summaries submitted as a PDF must be formatted with separate headings for the overview, statement on the
% intellectual merit of the proposed activity, and statement on the broader impacts of the proposed activity}. Failure
% to include these headings may result in the proposal being returned without review.
% Additional instructions for preparation of the Project Summary are available in FastLane.\\
% \subsection{Intellectual Merit} 
% The statement on intellectual merit should describe the potential of the proposed activity to advance knowledge.
% \subsection{Broader Impacts of the Proposed Work} 
% The statement on broader impacts should describe the potential of the proposed activity to benefit society and contribute to the achievement of specific, desired societal outcomes.

This document, submitted to the GNSS-SDR open-source software defined radio project as part of the 2018 Google Summer of Code (GSoC) program, proposes an algorithmic change to a key part of the signal processing pathway utilized by the GNSS-SDR project to perform outdoor positioning using a software defined radio. The document is comprised of several sections. The first section highlights the importance of advancement in the field of GNSS, including the importance of the GNSS-SDR project, and describes the motivation from an algorithmic standpoint for the implementation of the proposed change. The second section describes in detail the relevant techniques which represent the current state of the art for GNSS positioning, along with citations and literature supporting the effectiveness of these techniques in GNSS positioning. The second section also details the specific changes and additions that would need to be made to to the project software in order to implement those techniques. The third section details the academic and software design background of the author and the qualifications that the author has to contribute the described changes to the project. Finally, the fourth section of this document proposes the timeline and methodology by which these changes could be implemented within the scope of the 2018 GSoC program.

 % It is the belief of the author of this proposal that incoroporation of these state of the art techniques into the GNSS-SDR project will improve the performance of the software in conditions under which the algorithms and techniques currently utilized by the project perform either poorly or suboptimally in some way, and it is the hope of the author that the administration of the GNSS-SDR will agree with the merit of these techniques and support the author in this contribution to the project.