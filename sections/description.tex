\section{Project Description}
% \subsection{Introduction}
% \subsection{Proposed Study}
% The Project Description should provide a clear statement of the work to be undertaken and must include:
% objectives for the period of the proposed work and expected significance; relation to longer-term goals of the PI's
% project; and relation to the present state of knowledge in the field, to work in progress by the PI under other
% support and to work in progress elsewhere.
% 
% The Project Description should outline the general plan of work, including the broad design of activities to be
% undertaken, and, where appropriate, provide a clear description of experimental methods and procedures.
% Proposers should address what they want to do, why they want to do it, how they plan to do it, how they will
% know if they succeed, and what benefits could accrue if the project is successful. The project activities may be
% based on previously established and/or innovative methods and approaches, but in either case must be well
% justified. These issues apply to both the technical aspects of the proposal and the way in which the project may
% make broader contributions.
% 
% \subsection{Broader Impacts of the Proposed Work}
% The Project Description must contain, as a separate section within the narrative, a section labeled ``Broader
% Impacts of the Proposed Work". This section should provide a discussion of the broader impacts of the proposed
% activities. Broader impacts may be accomplished through the research itself, through the activities that are
% directly related to specific research projects, or through activities that are supported by, but are complementary to 
% the project. NSF values the advancement of scientific knowledge and activities that contribute to the
% achievement of societally relevant outcomes. Such outcomes include, but are not limited to: full
% participation of women, persons with disabilities, and underrepresented minorities in science, technology, engineering, and
% mathematics (STEM); improved STEM education and educator development at any level; increased public
% scientific literacy and public engagement with science and technology; improved well-being of individuals in
% society; development of a diverse,globally competitive STEM workforce; increased partnerships between
% academia, industry, and others; improved national security; increased economic competitiveness of the United
% States; and enhanced infrastructure for research and education.
% 
% \subsection{Results from Prior NSF Support}
% If any PI or co-PI identified on the project has received NSF funding (including any current
% funding) in the past five years, in formation on the award(s) is required,
% irrespective of whether the support was directly related to the proposal or not.
% In cases where the PI or co-PI has received more than one award (excluding amendments),
% they need only report on the one award most closely related to the proposal. Funding includes not just salary
% support, but any funding awarded by NSF. The following information must be provided:\\
% 
% \noindent
% \emph{\underline{Name of PI}}: NSF-Program (Award Number) ``Title of the Project'' (\$AMOUNT, PERIOD OF SUPPORT). 
% {\bf Publications:} List of publications resulting from the NSF award. A complete bibliographic citation for each
% publication must be provided either in this section or in the References Cited section of the proposal); if
% none, state: ``No publications were produced under this award.'' {\bf Research Products:} evidence of research products 
% and their availability, including, but not limited to: data, publications, samples, physical collections, software, 
% and models, as described in any Data Management Plan.

\subsection{Motivation}

\subsubsection{Overview}

Global Navigation Satellite Systems (GNSS) technology is accepted as the technology of choice for most outdoor position and time synchronization related applications. According to the fourth issue of the GNSS Market Report released by the European Global Satellite Navigation Systems Agency in 2015, it is projected that there will be around seven billion GNSS devices by 2019 \cite{GNSS_MR}. These devices are used in a multitude of different applications, ranging from navigation and network management systems for road, rail, aviation and maritime transportation, to to time synchronization applications in distributed systems such as power-grid management, agricultural management, surveying and communication. As more systems come to rely on GNSS for their operation, it is becoming increasingly important for GNSS technology to be as fast, reliable and accurate as possible in as many different conditions as possible, while remaining an affordable and easily distributively technology. One of the important ways in which GNSS technology may be able to advance to meet these increased demands for robustness is through the utilization of open source contribution and development projects such GNSS-SDR.

\subsubsection{Technical Motivation}

One of the critical steps of the GNSS receiver is carrier synchronization, where after the acquisition stage, the system must track the time-varying code delay, carrier phase and carrier Doppler frequency, which are used to correctly demodulate the navigation message and obtain the pseudoranges to the different satellites in view. While standard receivers rely on code-based positioning, carrier phase is of particular interest in modern carrier phase-based positioning techniques such as real-time-kinematic (RTK) and precise-point-positioning (PPP), used in high-precision GNSS receivers. Synchronization is a challenging task under harsh propagation conditions such as multipath, non-line-of-sight (NLOS), high-dynamics, shadowing, strong fadings or ionospheric scintillation. These non-nominal propagation conditions affect differently the code and carrier synchronization stages. While multipath and NLOS clearly impair code tracking, high-dynamics and ionospheric scintillation mainly affect carrier tracking. %In general, carrier synchronization tends to be more sensitive than time-delay tracking.

Standard mass-market GNSS receivers' synchronization rely on well-known delay-locked loop (DLLs), phase-locked loop (PLL) and frequency-locked loop (FLL) architectures, inherited from the analog era. The performance obtained with such techniques is good enough in benign propagation scenarios, but typically deliver poor performances under harsh propagation conditions. Both code and carrier tracking, or the joint code/carrier synchronization, can be formulated as an estimation problem, which can be solved using Bayesian filtering methods. The Kalman filter (KF) is the optimal analytical solution for linear/Gaussian systems, but suboptimal algorithms must be considered in nonlinear and/or non-Gaussian scenarios. The most popular solution for nonlinear/Gaussian systems is the extended KF (EKF), which linearizes the possibly nonlinear process or measurement functions and applies the standard linear KF equations. To avoid such linearization, which can be challenging in highly nonlinear scenarios, we can consider sigma-point Gaussian filters (SPGFs), or even more advanced techniques such as sequential Monte Carlo methods (a.k.a Particle Filters) in non-Gaussian scenarios. 

In the GNSS literature, it has been shown that KF-based synchronization solutions \cite{Vila17c} may overcome the performance limitations of standard approaches. The main advantages of the KF over DLL/PLL/FLL architectures are: i) it is formulated from an optimal filtering standpoint, ii) it has an implicit adaptive filter bandwidth \cite{Vila-14a}, and iii) a nonlinear KF (i.e., EKF or SPGF) allows to avoid the problems associated to the use of code, phase or frequency discriminators \cite{Vila-14b}. On the other hand, these filters are not as easy to tune and require more parameters than standard closed-loop techniques (e.g., process and measurement noise covariance matrices). One of the main practical disadvantages of KF-based techniques in the GNSS synchronization context, of particular interest in this contribution, is that their performance is typically shown only in controlled simulated scenarios, because their implementation and testing is more complex than using standard techniques (i.e., to use a standard PLL we only need to specify a few well-understood parameters such as the loop bandwidth or the damping factor). This typically prevents potential users to consider these techniques for real-life scenarios, and then it is not easy to judge their potential benefit for new applications. To improve GNSS availability in challenging propagation scenarios or its use for new scientific applications, it is of capital importance to be able to test the performance of such Bayesian filtering techniques with real signals and considering a complete receiver chain.

\subsection{Proposed Work}

Even with the advent of software defined radio (SDR) receivers in real-life applications, digital versions of traditional synchronization techniques are still commonly found in software implementations. This contribution takes advantage of the flexible and open architecture of GNSS-SDR, an open-source software-defined GNSS receiver project \cite{GNSS-SDR11}, to present the implementation and testing of complex KF architectures for both code and carrier tracking in real-life scenarios. The C++ implementation allows the use of highly-optimized matrix algebra libraries and high-performance Single Instruction Multiple Data (SIMD) operations that enable real-time operation of the receiver in multiple architectures, such as consumer-grade laptops and embedded computers. A freely available open-source implementation of synchronization Bayesian filtering techniques within an open-source GNSS receiver is expected to boost their use in new GNSS applications.

This contribution considers the following architectures: i) standard discriminator-based 2nd and 3rd order KF tracking, ii) adaptive KFs (AKFs), which adapt the measurement noise covariance to the actual working conditions, iii) joint code/carrier KF-based synchronization, and iv) nonlinear KFs operating with the input complex samples to avoid the use of discriminators, then making use of pilot signals. We present a fair performance comparison of these KF-based methods vs. DLL/PLL architectures using live GNSS signals, and its impact in high-level receiver metrics, such as the carrier phase tracking or code observables variance, and positioning precision when carrier smoothing techniques are applied. The implementation of the full software receiver, including the implementations used in this paper, will be freely available online at http://gnss-sdr.org. To the best of the authors' knowledge, this represents the first open-source implementation of KF-based GNSS synchronization. Moreover, this study opens the door for the integration of multiple sensors to further improve tracking performance, such as ultra-tight integration of Inertial Measurement Units (IMUs).



